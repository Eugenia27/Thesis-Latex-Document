% Chapter 1

\chapter{INTRODUCCI\'ON} % Main chapter title

\label{Chapter1} % For referencing the chapter elsewhere, use \ref{Chapter1} 

%----------------------------------------------------------------------------------------
% Define some commands to keep the formatting separated from the content 
\newcommand{\keyword}[1]{\textbf{#1}}
\newcommand{\tabhead}[1]{\textbf{#1}}
\newcommand{\code}[1]{\texttt{#1}}
\newcommand{\file}[1]{\texttt{\bfseries#1}}
\newcommand{\option}[1]{\texttt{\itshape#1}}

%----------------------------------------------------------------------------------------
%\section{}

%----------------------------------------------------------------------------------------

%\section{Learning \LaTeX{}}

%\subsection{A (not so short) Introduction to \LaTeX{}}

%\subsection{A Short Math Guide for \LaTeX{}}
%\enquote{Additional Documentation} section towards the bottom of the page.

%\subsection{Common \LaTeX{} Math Symbols}

%\url{http://www.sunilpatel.co.uk/latex-type/latex-math-symbols/}

%\subsection{\LaTeX{} on a Mac}

%----------------------------------------------------------------------------------------

%\section{Getting Started with this Template}

%\subsection{About this Template}

%----------------------------------------------------------------------------------------

%\section{What this Template Includes}

%\subsection{Folders}

%\keyword{Appendices} -- 

%\keyword{Chapters} -- this 
%\begin{itemize}
%\item Chapter 1: Introduction to the thesis topic
%\item Chapter 2: Background information and theory
%\item Chapter 3: (Laboratory) experimental setup
%\item Chapter 4: Details of experiment 1
%\item Chapter 5: Details of experiment 2
%\item Chapter 6: Discussion of the experimental results
%\item Chapter 7: Conclusion and future directions
%\end{itemize}
%This chapter layout is specialised for the experimental sciences, your discipline may be different.
%
%\keyword{Figures} -- 

%\subsection{Files}

%\keyword{example.bib} -- this 

%\keyword{MastersDoctoralThesis.cls} --

%\keyword{main.pdf} --

%\keyword{main.tex} -- 
%
%---------------------------------------------------------------------------------------
%\subsection{Tables}

%\begin{table}
%\caption{The effects of treatments X and Y on the four groups studied.}
%\label{tab:treatments}
%\centering
%\begin{tabular}{l l l}
%\toprule
%\tabhead{Groups} & \tabhead{Treatment X} & \tabhead{Treatment Y} \\
%\midrule
%1 & 0.2 & 0.8\\
%2 & 0.17 & 0.7\\
%3 & 0.24 & 0.75\\
%4 & 0.68 & 0.3\\
%\bottomrule\\
%\end{tabular}
%\end{table}


%\subsection{Figures}

%\begin{figure}[th]
%\centering
%\includegraphics{Figures/Electron}
%\decoRule
%\caption[An Electron]{An electron (artist's impression).}
%\label{fig:Electron}
%\end{figure}

%\subsection{Typesetting mathematics}
%This will produce Einstein's famous energy-matter equivalence equation:
%\begin{equation}
%E = mc^{2}
%\label{eqn:Einstein}
%\end{equation}

%All equations you write (which are not in the middle of paragraph text) are automatically given equation numbers by \LaTeX{}. If you don't want a particular equation numbered, use the unnumbered form:
%\begin{verbatim}
%\[ a^{2}=4 \]
%\%end{verbatim}

%----------------------------------------------------------------------------------------



